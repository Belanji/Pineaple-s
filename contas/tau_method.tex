\documentclass[amsmath,amsfonts,amssymb,superscriptaddress,showkeys,notitlepage,onecolumn]{revtex4-1}
\usepackage[utf8]{inputenc}
\usepackage{dcolumn}% Align table columns on decimal point
\usepackage{bm}% bold math
\usepackage{graphicx}
\usepackage{hyperref}
\usepackage{color}
\usepackage[caption=false]{subfig}


%Common used commands:

\newcommand{\tred}[1]{ \textcolor{red}{#1} }
\newcommand{\tblue}[1]{ \textcolor{blue}{#1} }

%Common used functions:

\newcommand{\dpartial}[1]{\ensuremath{\dfrac{\partial}{\partial #1}}}
\newcommand{\ddpartial}[1]{\ensuremath{\dfrac{\partial^2}{\partial #1^2}}}
\newcommand{\zint}[1]{ \ensuremath{  \int_{\tilde{z}=-1}^{\tilde{z}=1} #1 d\tilde{z} } }


\newcommand{\Npm}{\ensuremath{n_{\pm}(z,t)}}


\newcommand{\Np}{\ensuremath{n_{+}(\tilde{z},t)}}
\newcommand{\Nm}{\ensuremath{n_{-}(\tilde{z},t)}}
\newcommand{\V}{\ensuremath{V(\tilde{z},t)}}
\newcommand{\legP}[1]{\ensuremath{P_{#1}(\tilde{z})}}

\newcommand{\jpm}{\ensuremath{j_{\pm}(z,t)}}

% Constants:

\begin{document}
\title{PiNeaple- Poisson-Nerst-Planck: spectral version.}

\author{R.F. de Souza}

\begin{abstract} \tred{ Nothing much abstract to to talk about.} 
\end{abstract}

\maketitle

\section{Introduction:}



\section{Tau method:}

\subsection{Equations:}
\begin{equation}
  \dfrac{\partial}{\partial t}\Npm= -\dpartial{z} \jpm
\end{equation}
with
\begin{equation}\label{eq:StrongFlux}
  j_{\pm}(z,t)=-D_{\pm}\left[\dpartial{z} \Npm \pm \dfrac{q}{k_B T} \Npm \dpartial{z} V(z,t) \right] 
\end{equation}
and $-d \leq z \leq d$.

  \renewcommand{\Npm}{\ensuremath{n_{\pm}(\tilde{z},t)}}
  \renewcommand{\jpm}{\ensuremath{j_{\pm}(\tilde{z},t)}}

  
  Substituting $\tilde{z}=z/d$:
  \begin{equation}\label{eq:StrongPotential}
  \dfrac{\partial}{\partial t}\Npm= -\dfrac{1}{d}\dpartial{\tilde{z}} \jpm
\end{equation}
and  
\begin{equation}
  \jpm=-\tilde{D}_\pm\left[\dpartial{\tilde{z}}\Npm \pm \dfrac{q}{k_B T} \Npm \dpartial{\tilde{z}} V(\tilde{z},t) \right]
\end{equation}
with $\tilde{D}_\pm =D_\pm/d$ and $-1\leq z \leq 1$.

The boundary conditions is given by:
\begin{align}\nonumber
  j_{\pm}(\tilde{z}=-1,t)&=-\kappa_1 n_{\pm}(\tilde{z}=-1,t) + \dfrac{1}{\tau_1}\sigma_{\pm,1}(t)\\
  j_{\pm}(\tilde{z}=1,t)&=\kappa_2 n_{\pm}(\tilde{z}=1,t) -\dfrac{1}{\tau_2}\sigma_{\pm,2}(t)
\end{align}

The $\sigma_{\pm,1}$ and $\sigma_{\pm,2}$ obeys:
\begin{align}\nonumber
  \dfrac{\partial}{\partial t} \sigma_{\pm,1}(t)&=\kappa_1 n_{\pm}(\tilde{z}=-1,t)-\dfrac{1}{\tau_1} \sigma_{1,\pm}(t),\\
  \dfrac{\partial}{\partial t} \sigma_{\pm,2}(t)&=\kappa_2 n_{\pm}(\tilde{z}=1,t)-\dfrac{1}{\tau_2} \sigma_{2,\pm}(t),
\end{align}

The potential $V(\tilde{z},t)$ obeys the equation:
\begin{equation}
  \ddpartial{\tilde{z}} V(\tilde{z},t) =-\dfrac{q d^2}{\epsilon}(\Np-\Nm)
\end{equation}
with
\begin{equation}
  V\left(\pm 1 ,t \right)=\pm \dfrac{V_0}{2} \text{exp}(I \omega t)
\end{equation}

\subsection{Weighted residuals}

We are going to develop the functions $\Np$, $\Nm$ and $\V$ in terms o Legendre polynomials as following:

\begin{align}\nonumber
  \Np&=\sum_{i=0}^N  c^j_{+}(t) \legP{i}\\
  \Nm&=\sum_{i=0}^N  c^j_{-}(t) \legP{i}\\ \nonumber
  \V&=\sum_{i=0}^N   c^j_{v}(t) \legP{i}
\end{align}


Knowing that:
\begin{align}\nonumber
  \dpartial{\tilde{z}} \legP{i}&=\sum_{\substack{j=0 \\  j+i \; \text{odd}}}^{i-1} (2j+1) \legP{j}, \quad \text{and}\\
  \ddpartial{\tilde{z}} \legP{i}&=\sum_{\substack{j=0 \\  i+j \; \text{even}}}^{i-2} \left(j+\dfrac{1}{2} \right) \left(i(i+1)-j(j+1) \right) \legP{j},
\end{align}
we can write:
\begin{align}\label{eq:ddV}\nonumber
  \dpartial{z} \V&=\sum_{i=0}^{N} \sum_{\substack{j=0 \\  i+j \; \text{odd}}}^{i-1} (2j+1) {c^i}_v(t) \legP{j},\\
\ddpartial{z}  V&=\sum_{i=0}^{N} \sum_{\substack{j=0 \\  j+i \; \text{even}}}^{i-2} {c^i}_v(t) \left(j+\dfrac{1}{2} \right) \left(i(i+1)-j(j+1) \right)  \legP{j}
\end{align}


Writing the equations \eqref{eq:StrongFlux} and \eqref{eq:StrongPotential} as a residual:
\begin{align}\label{eq:Residual}\nonumber
  R_{\pm}(\tilde{z},t)&= \dpartial{t} \Npm + \dfrac{1}{d}\dpartial{\tilde{z}} \jpm\\
  R_{V}(\tilde{z},t)&= \ddpartial{\tilde{z}}  \V + \dfrac{q d^2}{\epsilon}(\Np-\Nm).
\end{align}

We will multiply the residuals by $P_i(\tilde{z})$ and integrate from $\tilde{z}=-1$ to $\tilde{z}=1$:
\begin{align}\nonumber \label{eq:residuals}
  \int_{\tilde{z}=-1}^{\tilde{z}=1} R_{\pm}(\tilde{z},t) P_k(\tilde{z}) d\tilde{z}&= 0\\
  \int_{\tilde{z}=-1}^{\tilde{z}=1} R_{V}(\tilde{z},t) P_k(\tilde{z}) d\tilde{z} &= 0
\end{align}
Taking $i=0,\ldots,N-2$. Taking 2 extra equations for the boundary conditions, we will have $3N+3$ equations for $3N+3$ variables.


\subsection{Manipulating Residuals equations:}

\subsubsection{Eletric potential equation.}

Expanding the eletric potential equation\eqref{eq:residuals}:
\begin{equation}\label{eq:vResiduals}
  \zint{ \ddpartial{z}  \V P_k(\tilde{z}) }+ \zint{ \dfrac{q d^2}{\epsilon}\left[\Np-\Nm \right] P_k(\tilde{z}) } =0 \quad \forall \quad i=0, \cdots, N-2
\end{equation}


Substituting \eqref{eq:ddV} into \eqref{eq:vResiduals}:
\begin{align}
  \zint{ \sum_{i=0}^{N} \sum_{\substack{j=0 \\  j+i \; \text{even}}}^{i-2}
  \left(j+\dfrac{1}{2} \right) \left(i(i+1)-j(j+1) \right) {c^i}(t)  \legP{j}
      P_k(\tilde{z})} 
  +\zint{ \dfrac{q d^2}{\epsilon} \sum_{i=0}^{N} [c^j_{+}(t)  - c^j_{-}(t)] P_k(\tilde{z}) P_i(\tilde{z}) } =0
\end{align}
for all $i=0,1,\cdots,N-2$.


The r.h.s can be simplified to:
\begin{align}\nonumber
  \zint{ \dfrac{q d^2}{\epsilon} \sum_{i=0}^{N} [c^i_{+}(t)  - c^i_{-}(t)] P_k(\tilde{z}) P_i(\tilde{z}) }&=-\dfrac{q d^2}{\epsilon} \sum_{i=0}^{N} [c^i_{+}(t)  - c^i_{-}(t)] \delta_{ik}\\
  &=\dfrac{q d^2}{\epsilon}  [c^k_{+}(t)  - c^k_{-}(t)]
\end{align}

The L.h.s:
\begin{align}\nonumber
  \zint{ \sum_{i=0}^{N} \sum_{\substack{j=0 \\  j+i \; \text{even}}}^{i-2}
  \left(j+\dfrac{1}{2} \right) \left(i(i+1)-j(j+1) \right) {c^i}(t)  \legP{j}
      P_k(\tilde{z})}  \\ \nonumber
  = \sum_{i=0}^{N} \sum_{\substack{j=0 \\  j+i \; \text{even}}}^{i-2}
  \left(j+\dfrac{1}{2} \right) \left[i(i+1)-j(j+1) \right] {c^i}(t)\gamma_k \delta_{jk}
\end{align}

Therefore we have:
\begin{align}
\sum_{i=0}^{N} \sum_{\substack{j=0 \\  j+i \; \text{even}}}^{i-2}
  \left(j+\dfrac{1}{2} \right) \left[i(i+1)-j(j+1) \right] {c^i}(t)\gamma_k \delta_{jk}=-\dfrac{q d^2}{\epsilon}  [c^k_{+}(t)  - c^k_{-}(t)] \quad \forall \quad k=0,\cdots, N-2.
\end{align}

The boundary conditions are:
\begin{equation}\label{eq:expandedVboundaries}
 \sum_{i=0}^N {c^i_{v}(t)} P_i(\pm 1)=\pm \dfrac{V_0}{2} \text{exp}(I \omega t).
\end{equation}

Knowing that $P_i(\pm 1)=(\pm 1)^i$, we can divide \eqref{eq:expandedVboundaries} into:
\begin{align}\nonumber
  \sum_{i=0}^N {c^i_{v}(t)}= \dfrac{V_0}{2} \text{exp}(I \omega t),\\
  \sum_{i=0}^N (-1)^i {c^i_{v}(t)}= -\dfrac{V_0}{2} \text{exp}(I \omega t).\\
\end{align}

\subsubsection{Concentration equations:}


Writing the equations \eqref{eq:StrongFlux} and \eqref{eq:StrongPotential} as a residual:
\begin{align}\label{eq:Residual}\nonumber
  R_{\pm}(\tilde{z},t)&= \dpartial{t} \Npm-\dfrac{\tilde{D}_\pm}{d} \dpartial{\tilde{z}}\left[\dpartial{\tilde{z}}\Npm \pm \dfrac{q}{k_B T} \Npm \dpartial{\tilde{z}} V(\tilde{z},t) \right]\\
  &=\dpartial{t} \Npm-\dfrac{\tilde{D}_\pm}{d}\left\lbrace\ddpartial{\tilde{z}}\Npm \pm \dfrac{q}{k_B T} \left[\dpartial{\tilde{z}} \Npm \dpartial{\tilde{z}} V(\tilde{z},t) + \Npm \ddpartial{\tilde{z}} V(\tilde{z},t)\right] \right\rbrace
\end{align}

\begin{align}\label{eq:Residual}\nonumber
\zint{ \dpartial{t} \Npm P_i(\tilde{z})} - \zint{ \dfrac{\tilde{D}_\pm}{d}\left\lbrace\ddpartial{\tilde{z}}\Npm \pm \dfrac{q}{k_B T} \left[\dpartial{\tilde{z}} \Npm \dpartial{\tilde{z}} V(\tilde{z},t) + \Npm \ddpartial{\tilde{z}} V(\tilde{z},t)\right] \right\rbrace P_i(\tilde{z})}=0
\end{align}

Going by parts:
\begin{equation}
\zint{ \dpartial{t} \Npm P_k(\tilde{z}) d\tilde{z}}= \dpartial{t} c^k_{\pm}(t)\gamma_k 
\end{equation}

\begin{align}\nonumber
  \zint{\dfrac{\tilde{D}_\pm}{d} \ddpartial{\tilde{z}} \Npm P_{k}(\tilde{z}) }&=\sum_{i=0}^{N} \sum_{\substack{j=0 \\  i+j \; \text{even}}}^{i-2} \zint{
  \left(j+\dfrac{1}{2} \right) \left(i(i+1)-j(j+1) \right)
  {c^i}_{\pm}(t)  P_j(\tilde{z}) P_k(\tilde{z}) },\\
  &=\sum_{i=0}^{N} \sum_{\substack{j=0 \\  i+j \; \text{even}}}^{i-2}  \dfrac{\tilde{D}_\pm}{d} \left(j+\dfrac{1}{2} \right) \left(i(i+1)-j(j+1) \right)
  {c^i}_{\pm}(t) \gamma_k \delta_{jk}
\end{align}

\begin{equation}
\zint{\legP{k} \legP{j}   \legP{i}  }= \dfrac{2}{2k+1} Cg(i,j,k,0,0,0)^2
\end{equation}


Yet another part:


\begin{align}\nonumber
  \zint{ \dfrac{\tilde{D}_\pm q}{k_B T d}\dpartial{\tilde{z}} \Npm \dpartial{\tilde{z}} V(\tilde{z},t) \legP{k} }=\\
=  \dfrac{\tilde{D}_\pm q}{k_B T d} \zint{ \sum_{i=0}^{N} \sum_{p=0}^{N} \sum_{\substack{j=0 \\  j+i \; \text{odd}}}^{i-1}    \sum_{\substack{q=0 \\  p+q \; \text{odd}}}^{i-1} (2j+1)(2q+1) c^i_{\pm}(t)  c^p_{v}(t) \legP{j} \legP{q}   \legP{k}  }\\
  =
  \dfrac{\tilde{D}_\pm q}{k_B T d}  \sum_{i=0}^{N} \sum_{p=0}^{N} \sum_{\substack{j=0 \\  j+i \; \text{odd}}}^{i-1}    \sum_{\substack{q=0 \\  p+q \; \text{odd}}}^{i-1} 2(2j+1) c^i_{\pm}(t)  c^p_{v}(t) Cg(j,k,q,0,0,0)^2
\end{align}

And the final part:
\begin{align}\nonumber
  \zint{\dfrac{D q}{k_B T d}  \Npm \ddpartial{\tilde{z}} V(\tilde{z},t)}&=\\\nonumber
  =  \sum_{i=0}^{N} \sum_{p=0}^{N}  \sum_{\substack{q=0 \\  p+q \; \text{even}}}^{i-2} \zint{ \dfrac{D q}{k_B T d}      c^i_{\pm}(t)  c^p_{v}(t)  \left(q+\dfrac{1}{2} \right) \left[p(p+1)-q(q+1) \right] \legP{i} \legP{q} \legP{k}}\\
  =\sum_{i=0}^{N} \sum_{p=0}^{N}  \sum_{\substack{q=0 \\  p+q \; \text{even}}}^{i-2}  \dfrac{D q}{k_B T d}      c^i_{\pm}(t)  c^p_{v}(t)  \dfrac{2 \left(q+\dfrac{1}{2} \right) \left[p(p+1)-q(q+1) \right]}{2q+1}  Cg(i,k,q,0,0,0)^2
\end{align}

Here goes the boundary conditions:
\begin{align}
  -\tilde{D}_\pm\left[\dpartial{\tilde{z}}\Npm \pm \dfrac{q}{k_B T} \Npm \dpartial{\tilde{z}} V(\tilde{z},t) \right]+\kappa_1 n_{\pm}(\tilde{z}=1,t) -\dfrac{1}{\tau_1}\sigma_{\pm,1}(t)=0\\
  -\tilde{D}_\pm\left[\dpartial{\tilde{z}}\Npm \pm \dfrac{q}{k_B T} \Npm \dpartial{\tilde{z}} V(\tilde{z},t) \right]-\kappa_2 n_{\pm}(\tilde{z}=-1,t) +\dfrac{1}{\tau_2}\sigma_{\pm,2}(t)=0
\end{align}

Knowing that
\begin{align}
  P_{i}(\tilde{z}=\pm 1)&=(\pm 1)^i\\
  \dpartial{\tilde{z}}P_{i}(\tilde{z}=\pm 1)&=\dfrac{1}{2} (\pm 1)^{(i-1)} i (i+1),
\end{align}
we have
\begin{equation}
  \dfrac{\partial}{\partial \tilde{z}} V(\tilde{z}=\pm 1,t)=\sum_{i=0}^{N} \dfrac{1}{2} (\pm 1)^{i-1} i(i+1) c_{v}^i(t) 
\end{equation}

\begin{align}
  \sum^N_{i=0} \left[-\tilde{D}_\pm \dfrac{1}{2} (-1)^{(i-1)} i (i+1) +(-1)^{i} (\kappa_1 \mp \dfrac{q}{k_B T } \dfrac{\partial}{\partial \tilde{z}} V(\tilde{z}=- 1,t) )\right]c^i_{\pm}(t) - \dfrac{1}{\tau_1}\sigma_{1,\pm}=0\\
  \sum^N_{i=0} \left[-\tilde{D}_\pm \dfrac{1}{2}  i (i+1) \mp \dfrac{q}{k_B T} \dfrac{\partial}{\partial \tilde{z}} V(\tilde{z}=- 1,t)- \kappa_2  \right] c^i_{\pm}(t) +\dfrac{1}{\tau_2}\sigma_{2,\pm}=0
\end{align}

\subsubsection{Sigma equations:}

Knowing that:
\begin{align}
  \kappa_1 n_{\pm}(\tilde{z}=1,t)=\sum_{i=0}^N\kappa_1 c_{\pm}^i(t),\\
  \kappa_2 n_{\pm}(\tilde{z}=-1,t)=\sum_{i=0}^N\kappa_2 (-1)^{i} c_{\pm}^i(t),
\end{align}
we have:

The $\sigma_{\pm,1}$ and $\sigma_{\pm,2}$ obeys:
\begin{align}\nonumber
  \dfrac{\partial}{\partial t} \sigma_{\pm,1}(t)&=\kappa_1 \sum^{N}_{i=0} (-1)^i c^i_{\pm}(t)-\dfrac{1}{\tau_1} \sigma_{1,\pm}(t),\\
  \dfrac{\partial}{\partial t} \sigma_{\pm,2}(t)&=\kappa_2 \sum^{N}_{i=0}  c^i_{\pm}(t)-\dfrac{1}{\tau_1} \sigma_{2,\pm}(t),
\end{align}


\end{document}






